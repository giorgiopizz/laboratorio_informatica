\documentclass{article}
% Allow the usage of utf8 characters
\usepackage[utf8]{inputenc}
% Uncomment the following line to allow the usage of graphics (.png, .jpg)
%\usepackage{graphicx}

% Start the document
\begin{document}

% Create a new 1st level heading
\section*{Primo principio della termodinamica}
Il primo principio della termodinamica sancisce la conservazione dell'energia. Dalla dinamica è noto che $W_e=\Delta K$ se vale che $\vec{F}_e=- \vec{F}_c$, dove $\vec{F}_c$ è una forza conservativa, allora $W_e=\Delta U_p$. \\
Considerando ora tre casi recipienti \textit{adiabatici} contenenti tutti acqua ad una temperatura iniziale $T_1$. Nel primo recipiente è contenuto un molinello che ruotando riscalda l'acqua, nel secondo è contenuta una resistenza percorsa da corrente che riscalda l'acqua per effetto Joule e nel terzo due piastre che vengono fatte strisciare l'una contro l'altra sfruttando l'attrito per riscaldare.  


\end{document}