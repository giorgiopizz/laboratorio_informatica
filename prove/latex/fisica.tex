\documentclass{article}
% Allow the usage of utf8 characters
\usepackage[utf8]{inputenc}
% Uncomment the following line to allow the usage of graphics (.png, .jpg)
%\usepackage{graphicx}

% Start the document
\begin{document}

% Create a new 1st level heading
\section{Variazioni di pressione e temperatura}
Preso un recipiente con pistone mobile contenente un vapore saturo, ovvero l'insieme di vapore e liquido di uno stesso composto, se si espande il pistone, il liquido evaporerà mantendo la pressione costante fino a quando tutto il liquido sarà evaporato. Dopo di che il cilindro contente solo vapore si comporterà similmente ad un gas, diminuendo la pressione. Se successivamente si comprimesse il cilindro tramite il pistone il vapore condeserebbe mantenendo la pressione costante. Una volta evaporato tutto, il liquido si comporterebbe similmente ad un gas aumentando la sua pressione. Le curve rappresentanti la pressione e il volume del composto sono dipendenti dalla temperatura, a temperature più basse si ha l'isobara in cui coesistono vapore e liquido più lunga(ovvero bisogna comprimere o espandere il composto sempre di più per farlo codensare o evaporare tutto). Per temperature più alte della temperatura critica(la temperatura per cui non coesistono vapore e liquido) le curve corrispondo a delle iperboli.
Data l'equazione ${(\frac{\partial U}{\partial V})}_T=T{(\frac{\partial U}{\partial T})}_V-p$ e posti $m_1$, $u_1$ e $v_1$ per il liquido, come rispettivamente la massa, l'energia interna per unità di massa e il volume per unità di massa(il reciproco della densità) e $m_2$ $u_2$ e $v_2$ per il gas si trova che $M=m_1+m_2$ da cui 
\begin{equation}
V(T)=m_1 v_1(
T)+m_2 v_2(T)
\end{equation}
\begin{equation}
U(T)=m_1 u_1(T)+m_2 u_2(T)
\end{equation} 

% Uncomment the following two lines if you want to have a bibliography
%\bibliographystyle{apalike}
%\bibliography{document}

\end{document}