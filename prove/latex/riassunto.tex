\documentclass{article}
% Allow the usage of utf8 characters
\usepackage[utf8]{inputenc}
% Uncomment the following line to allow the usage of graphics (.png, .jpg)
%\usepackage{graphicx}
% Start the document
\begin{document}

% Create a new 1st level heading
\section*{Metodi di calcolo per Termodinamica}
\subsection*{Calorimetro}
Nel calorimetro si ha sempre che $Q=0$ ovvero non si ha alcuno scambio di calore con l'esterno, $\sum Q_i=0$. La variabile fondamentale in queste tipologie di esercizio è $T$, solitamente è richiesto di trovare la temperatura di equilibrio $T_e$, infatti lo scambio di calore interno al calorimetro si ha solo quando le temperature sono diverse: nel momento in cui tutti i corpi hanno temperatura uguale si raggiunge l'equilibrio e quindi $T_e$. Possono verificarsi dei cambiamenti di fase dunque vi sarà il calore latente $\lambda$. I parametri possono essere le masse $m$ e i loro calori specifici $c_s$ o il numero di moli $n$ e il calore specifico molare $c_m$ oppure solo la capacità termina. 

\subsection*{Trasformazioni di gas}
Per i gas vale l'equazione di stato $pV=nRT$ e $U$ dipende solo dalla temperatura. È bene anche ricordare la legge di Dalton: \textit{La pressione esercitata da una miscela di gas è uguale alla somma delle pressioni parziali di tutti i gas che compongono la miscela}. Nelle trasformazioni più comuni, salvo casi in cui è specificato diversamente, il gas è racchiuso dentro ad un contenitore che può essere adiabatico, diatermico, con pistone mobile e messo a contatto con diverse sorgenti termiche. Si dà per noto che $c_p=c_v+R$, $\gamma=\frac{c_p}{c_v}$ e che per gas monoatomici $c_v=3/2$ mentre per gas biatomici $c_v=5/2$.
\subsubsection*{Isoterme}
Essendo $T_{in}=T_{fin}$ si deduce che $\Delta U=0$, per cui il primo principio della termodinamica diventa $Q=L$. Per le isoterme vale pV=costante. Il lavoro in una trasformazione reversibile è $W=nRT\ln(\frac{V_2}{V_1})$ anche se è sempre meglio analizzare il caso in esame prima di usare la formula appena riportata. Il calore è uguale al lavoro e la variazione di entropia per un gas durante un'isoterma reversibile è $\Delta S=\int \frac{\delta Q}{T}=nR\ln({\frac{V_2}{V_1}})$
\subsubsection*{Isocore}
Nelle trasformazioni isocore il volume rimane costante quindi per un gas in un contenitore il lavoro è nullo quindi $\Delta U=Q$. Vale che $Q= nc_v\Delta T$ e infatti in un'isocora $\Delta U=nc_v\Delta T$. La variazione di entropia per un gas durante un'isocora è $\Delta S=nc_v\ln({\frac{T_2}{T_1}})$. Le isocore irreversibili si hanno quando ad esempio il gas è posto a contatto con una sorgente a temperatura $T_2$. 

\subsubsection*{Isobare}
Nelle isobare la pressione esterna è uguale a quella del gas quindi $p=p_e$. Ne segue che il lavoro è $W=p(V_2-V_1)$. Il calore ceduto o assorbito dal gas durante la trasformazione è $Q=nc_p\Delta T$.Essendo la variazione di energia interna per un gas che passa da $T_1$ a $T_2$ indipendente dalla trasformazione che segue in quanto $U$ per i gas dipende solo dalla temperatura, si ha che $\Delta U_{isobara}= \Delta U_{isocora}$ ovvero $\Delta U_{isobara}=nc_v\Delta T$. La variazione di entropia per il gas durante un'isobara reversibile è $\Delta S=nc_p\ln({\frac{T_2}{T_1}})$ o equivalentemente $\Delta S=nc_p\ln({\frac{V_2}{V_1}})$. Le isobare irreversibili si hanno, come le isocore, quando il gas è posto a contatto con una sorgente a temperatura $T_2$.
\subsubsection*{Adiabatiche}

\subsubsection*{Espansione libera}
L'espansione liberà si ha quando un gas, inizialmente contenuto in una porzione di contenitore, viene lasciato libero di occupare l'intero volume del recipiente.
Nelle espansioni libere il lavoro che il gas compie verso l'esterno è nullo. Se l'espansione avviene all'interno di un contenitore adiabatico anche il calore scambiato verso l'esterno è nullo e ne segue che $\Delta U=0$ ovvero $\Delta T=0$. Essendo una trasformazione irreversibile $\Delta S_{sist}>0$ ossia $\Delta S=\Delta S_{isoterma}$.
\\Se due o più gas sono inizialmente contenuti in un contenitore adiabatico e separati da barriere, una volta rimosse le barriere formano una miscela di gas. Non si tratta di espansione libera ma ne condivide alcune proprietà. Il lavoro verso l'esterno è nullo, come il calore scambiato verso l'esterno($W=0$,$Q=0$) ne segue che $\Delta U_{sist}=0$ tuttavia la temperatura dei singoli gas è soggetta a variazione. L'equilibrio si ha quando tutti i gas raggiungono la $T_e$, temperatura che si trova ponendo la somma delle variazioni di energia interna uguale a 0 ($\sum (\Delta U_i)=0$). In questo caso la trasformazione è reversibile quindi $\Delta S_u=0$, $\Delta S_{sist}=\sum(\Delta S_i)$ con $\Delta S_i=\int \frac{dU+pdV}{T}$ bisogna porre attenzione al fatto che pur non compiendo lavoro verso l'esterno ogni gas compie lavoro sugli altri gas, la somma di questi lavori è nulla ma nell'entropia non si semplificano.


\subsubsection*{Trasformazioni miste}
Per le trasformazioni che non appartengono a nessuna delle categorie precedenti viene fornita o la pressione in funzione del volume $p(V)$ oppure la temperatura in funzione dell'entropia $T(S)$, quindi sfruttando il primo principio è possibile ricavare le informazioni necessarie. 
\subsection*{Cicli}
Nei cicli, essendo $U$ e $S$ funzioni di stato, $\Delta U=0$ e $\Delta S_{gas}=0$ in quanto stato iniziale e finale coincidono

\end{document}
