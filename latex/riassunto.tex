\documentclass{article}
% Allow the usage of utf8 characters
\usepackage[utf8]{inputenc}
% Uncomment the following line to allow the usage of graphics (.png, .jpg)
%\usepackage{graphicx}

% Start the document
\begin{document}

% Create a new 1st level heading
\section*{Metodi di calcolo per Termodinamica}
\subsection*{Calorimetro}
Nel calorimetro si ha sempre che $Q=0$ ovvero non si ha alcuno scambio di calore con l'esterno, $\sum Q_i=0$. La variabile fondamentale in queste tipologie di esercizio è $T$, solitamente trovare la temperatura di equilibrio $T_e$, infatti lo scambio di calore interno al calorimetro si ha solo quando le temperature sono diverse: nel momento in cui tutti i corpi hanno temperatura uguale si raggiunge l'equilibrio e quindi $T_e$. Possono verificarsi dei cambiamenti di fase dunque vi sarà il calore latente $\lambda$. I parametri possono essere le masse $m$ quindi dei calori specifici $c_s$ o il numero di moli $n$ e il calore specifico molare $c_m$ oppure solo la capacità termina. 

\subsection*{Trasformazioni di gas}
Per i gas vale l'equazione di stato $pV=nRT$ e $U$ dipende solo dalla temperatura. È bene anche ricordare la legge di Dalton: \textit{La pressione esercitata da una miscela di gas è uguale alla somma delle pressioni parziali di tutti i gas che compongono la miscela}
\subsubsection*{Isoterme}
Essendo $T_{in}=T_{fin}$ si deduce che $\Delta U=0$, per cui il primo principio della termodinamica diventa $Q=L$. Per le isoterme vale pV=costante. Il lavoro in una trasformazione reversibile è $W=nRT\ln(\frac{V_2}{V_1})$ anche se è sempre meglio analizzare il caso in esame prima di usare la formula appena riportata. Il calore è uguale al lavoro e l'entropia per un'isoterma reversibile è $\Delta S=\int \frac{\delta Q}{T}=nR\ln{\frac{V_2}{V_1}}$
\subsubsection*{Isocore}
\subsubsection*{Isobare}
\subsubsection*{Adiabatiche}
\subsubsection{Espansione libera}

\subsection*{Cicli}


\end{document}